\documentclass[a4paper,11pt]{article}
\usepackage{amsmath, amsthm, amssymb}
\usepackage{bm}
\usepackage{hyperref}

% Theorem environments
\newtheorem{axiom}{Axiom}
\newtheorem{definition}{Definition}

% Category symbols (for consistency, although not strictly needed here)
\newcommand{\catG}{\mathscr{G}}

\begin{document}

\title{W-Gradient Axioms and Severance Distance\\
\small Formal Definition of Ethical Discontinuity Hierarchy in GSE}
\author{GSE Formal Foundation}
\date{}
\maketitle

\section*{1. 序論:断絶勾配の公理化}

本セクションでは、Principle W(客観的重み付け原則)を、倫理的断絶の深刻度を測る構造公理系として再定義する [2], [3]。これにより、RBA(責任的行為調整)プロトコルは、断絶の深刻度に応じた構造的複雑性 \(C\) の調整量 \(\Delta C\) を決定する論理的基盤を得る [4], [5]。

\section*{2. W-Gradient Axioms (Ax.W1 - Ax.W5)}

Principle W1からW5は、Severance の深刻度を測るSeverance Distance \(d_S\) を定義するための公理として機能する [3]。

\begin{axiom}[Ax.W1: Universal Continuity Axiom]
あらゆる I-Node に共通する普遍的基盤(生命、生態系の存続など)への断絶は、常に絶対的に回避されなければならない [6], [1]。
\end{axiom}

\begin{axiom}[Ax.W2: Collective System Axiom]
共同体・集合的システム(国家、広範な経済基盤など)の維持に関わる構造的連関は、Ax.W1に次いで高い保護度を有する [6], [7]。
\end{axiom}

\begin{axiom}[Ax.W3: Historical Integrity Axiom]
長期間培われ最適化された継時的波長(歴史的連続性、積層構造)を担う I-Node の固有性は、恣意的な破却により不可逆断絶となる [6], [7]。
\end{axiom}

\begin{axiom}[Ax.W4: Specialized Competence Axiom]
代替不可能な個人の特殊性・能力、または集団の根源的波長(固有性)の喪失は、I-Node の局所的崩壊を引き起こす断絶となる [6], [7]。
\end{axiom}

\begin{axiom}[Ax.W5: Minimal Disruption Axiom]
断絶が限定的であり、かつ回復が比較的容易なローカルな可逆的パタンの破壊は、最も浅い階層に属する [6], [7]。
\end{axiom}

\section*{3. Severance Distance と RCD 調整勾配}

\subsection{定義 (Severance Distance $d_S$)}
倫理的断絶の深刻度 $d_S$ は、Ax.W1からAx.W5に対応する離散的な距離関数として定義される [4], [3]。

\begin{definition}[Severance Distance $d_S$]
倫理的断絶が引き起こす深刻度を測る Severance Distance $d_S$ は、Ax.W1からAx.W5の階層に従い、次の離散値を取る [4], [3]。
\[
d_S \in \{0, 1, 2, 3, 4\}
\]
\end{definition}

\begin{itemize}
    \item $d_S = 0$ は Ax.W1 違反(普遍的基盤の破壊)に対応し、最も回避すべき断絶である [5], [3]。
    \item $d_S = 4$ は Ax.W5 違反(ローカルな可逆的断絶)に対応し、最も浅い階層である [5], [3]。
\end{itemize}
距離が小さいほど、回避すべき倫理勾配が大きいことを示す [4], [3]。

\subsection{W-weighted Gradient (RBA への適用)}
RBAによる構造的複雑性 \(C\) の調整量 \(\Delta C\) は、Severance Distance \(d_S\) に依存し、Severance の深刻度に応じて \(C\) の増大を強制する [4], [5], [8]。

\[
\Delta C = \alpha \cdot \frac{1}{d_S + \epsilon} \tag{3.1}
\]
ここで、\(\alpha\) は調整係数、\(\epsilon\) は \(d_S=0\) でのゼロ除算を回避するための微小量である [4], [5], [8]。

Ax.W1 違反($d_S=0$)が発生した場合、この数式により RBA は**最大の調整勾配**を適用し、**構造的複雑性 $C$ の急速な増大(RCD)**を強制する [5], [8]。これにより、Severance の深刻度に応じた**不可逆性の担保**が強化される [5], [8]。

\end{document}
