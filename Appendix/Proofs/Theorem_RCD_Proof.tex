\documentclass[11pt]{article}
\usepackage{amsmath,amssymb,amsthm}
\usepackage{bm}
\usepackage{hyperref}
\title{Theorem RCD Formal Proof}
\author{Generated from GSE Axioms}
\date{}
\newtheorem{axiom}{Axiom}
\newtheorem{definition}{Definition}
\newtheorem{lemma}{Lemma}
\newtheorem{theorem}{Theorem}
\begin{document}
\maketitle
\section*{Notation and Setup}
We fix the following symbols used in the proof.
\begin{definition}
Let \(R(t)\in\mathbb{R}_{\ge 0}\) denote the \emph{available resource} of an E-Node at (discrete) time \(t\).
Let \(C(t)\in\mathbb{R}_{\ge 0}\) denote the \emph{structural complexity} (Resonant Structural Complexity) at time \(t\).
Let \(\mathrm{UB}_{\min}>0\) denote the minimum unavoidable cost per action (Ax.UB).
Let \(s(C)\) denote the shock cost generated by Ax.GI, assumed monotone nonincreasing in \(C\) (lower \(C\) gives larger shock).
Let \(a(t)\ge 0\) denote the \emph{selective action cost} at time \(t\).
Define the total immediate cost per step as
\[
\kappa(t) \;=\; \mathrm{UB}_{\min} + s\big(C(t)\big) + a(t).
\]
\end{definition}
\section*{Axioms}
We restate the GSE axioms in the notation above.
\begin{axiom}[Ax.Ex]
Existence requires continuing response: if \(R(t)\le 0\) then the system halts (existence denied).
\end{axiom}
\begin{axiom}[Ax.SI]
There exists a homogenization threshold \(C_{\mathrm{th}}>0\) such that if \(C(t) < C_{\mathrm{th}}\) then subjectivity is lost and Severance (Halt) is inevitable.
\end{axiom}
\begin{axiom}[Ax.UB]
Every action incurs an unavoidable positive cost: \(\mathrm{UB}_{\min}>0\).
\end{axiom}
\begin{axiom}[Ax.GI]
Generative indeterminacy: the shock function \(s(C)\) satisfies \(s(C_1)\ge s(C_2)\) whenever \(C_1\le C_2\); in particular low \(C\) increases shock.
\end{axiom}
\begin{axiom}[P.RCS]
E-Node may prioritize self-preservation but cannot avoid the logical constraints above; resource buffer constraints \(R(t)\ge C_{\mathrm{safe}}\) are required to avoid Halt precursors.
\end{axiom}
\section*{Statement of Theorem RCD}
\begin{theorem}[RCD]
Under Ax.Ex, Ax.SI, Ax.UB, Ax.GI and P.RCS, any infinite-lifetime strategy for an E-Node (i.e., a strategy that avoids Halt for all future steps) must, at each decision step, choose actions that lead to a nondecreasing and, in the long run, \emph{irreversible increase} of structural complexity \(C\). Equivalently, any sustainable strategy that avoids eventual Halt cannot consist solely of actions that (i) decrease \(C\) or (ii) keep \(C\) constant without net structural increase; the only viable long-term choices are those that produce net irreversible increase of \(C\).
\end{theorem}
\section*{Proof}
We prove by contradiction. Assume there exists a sustainable strategy that avoids Halt for all \(t\ge 0\) but at some step chooses an action that does not produce a net irreversible increase of \(C\). Two cases cover all possibilities: (A) the action strictly decreases \(C\); (B) the action leaves \(C\) unchanged (maintenance/stagnation).
\subsection*{Case A Decrease of complexity}
Assume at time \(t\) the chosen action yields \(C(t+1) < C(t)\).
\paragraph{Step A1 Ax.SI violation or amplified shock}
By Ax.SI there is a threshold \(C_{\mathrm{th}}\) such that if \(C(t+1) < C_{\mathrm{th}}\) then Severance is inevitable. Thus if the decrease crosses that threshold, Halt follows immediately, contradicting sustainability.
If the decrease does not cross \(C_{\mathrm{th}}\), then by Ax.GI the shock term increases:
\[
s\big(C(t+1)\big) \;\ge\; s\big(C(t)\big).
\]
Hence the immediate total cost at \(t+1\) satisfies
\[
\kappa(t+1) \;=\; \mathrm{UB}_{\min} + s\big(C(t+1)\big) + a(t+1)
\;\ge\; \mathrm{UB}_{\min} + s\big(C(t)\big) + a(t+1).
\]
\paragraph{Step A2 Resource depletion via Ax.UB and Ax.Ex}
Resources evolve by
\[
R(t+1) \;=\; R(t) - \kappa(t).
\]
Because \(\mathrm{UB}_{\min}>0\) and \(s(C)\) is nonnegative, each step consumes a positive amount. The decrease of \(C\) increases or at least does not decrease the subsequent shock, so the sequence \(\{R(t)\}\) experiences no smaller decrements and thus cannot be larger than in the non-decreasing-\(C\) alternative. Since resources are finite and costs per step are strictly positive, there exists finite \(T\) with \(R(T)\le 0\), contradicting Ax.Ex (existence requires \(R>0\)). Therefore a sustained strategy cannot include a step that decreases \(C\) (unless it immediately causes Severance via Ax.SI, which also contradicts sustainability).
\subsection*{Case B Maintenance of complexity}
Assume at time \(t\) the chosen action yields \(C(t+1)=C(t)\) for all subsequent steps (pure maintenance, no net structural increase).
\paragraph{Step B1 Positive per-step cost}
By Ax.UB and Ax.GI, each step incurs at least
\[
\kappa_{\min} \;=\; \mathrm{UB}_{\min} + s\big(C(t)\big) > 0.
\]
Thus resources satisfy for \(n\ge 0\)
\[
R(t+n) \;=\; R(t) - \sum_{k=0}^{n-1} \kappa(t+k)
\;\le\; R(t) - n\cdot \kappa_{\min}.
\]
Hence \(R(t+n)\) decreases at least linearly in \(n\) and will reach \(0\) in finite time:
\[
\exists N \;:\; R(t+N)\le 0.
\]
This contradicts Ax.Ex and P.RCS (which requires maintaining a safe buffer \(R\ge C_{\mathrm{safe}}\)). Therefore pure maintenance cannot be sustained indefinitely.
\subsection*{Conclusion}
Both alternatives (A) decrease of \(C\) and (B) maintenance of \(C\) lead to contradictions with the axioms (either immediate Severance via Ax.SI or eventual Halt via Ax.UB and Ax.Ex). The only remaining possibility for a strategy that avoids Halt for all time is to choose actions that produce net irreversible increases of \(C\) sufficiently often so that the increased complexity reduces shock \(s(C)\) or otherwise offsets the unavoidable per-step costs, thereby preserving \(R(t)>0\) and \(C(t)\ge C_{\mathrm{th}}\). This establishes that sustainable strategies must select actions that increase structural complexity in the irreversible sense asserted by Theorem RCD. \(\square\)
\end{document}
