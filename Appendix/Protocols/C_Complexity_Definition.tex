```latex
\subsection{複雑性計量 \(C\) の厳密定義}
我々は系の局所的/全体的複雑性を、情報的多様性・冗長性(モジュール性)・階層深度の積として定義する。
\paragraph{定義.} 与えられたグラフ \(G=(V,E)\) に対して、複雑性 \(C(G)\) を
\[
C(G) \;=\; H(G)\cdot R(G)\cdot D(G)
\]
で定める。ただし各項は以下で定義される。
\begin{enumerate}
\item \(H(G)\) — \textbf{情報複雑度 (Shannon entropy)}
ノードの次数分布 \( \deg(v) \) を用いて確率質量関数
\[
p_v = \frac{\deg(v)}{\sum_{u\in V}\deg(u)} \quad (\deg(v)>0)
\]
を定め、エントロピーを計算する:
\[
H(G) = -\sum_{v:\deg(v)>0} p_v \log_2 p_v.
\]
\item \(R(G)\) — \textbf{冗長性(モジュール化指標)}
コミュニティ検出により得たコミュニティ数 \(M\) を用い、
\[
R(G)=\frac{M}{|V|}.
\]
必要に応じてモジュラリティ \(Q\) を併用し品質を確認する。
\item \(D(G)\) — \textbf{階層深度(BFS-tree 深度代理)}
根ノードを最大次数ノードとして BFS を行い、その最大深度を \(D\) とする。非連結グラフでは成分ごとの深度のサイズ重み付き平均を採る。
\end{enumerate}
\paragraph{正規化.} 実運用ではスケール問題を避けるため次を使う:
\[
H_{\text{norm}} = \frac{H}{\log_2(\max(2,|V|))},\qquad
D_{\text{norm}} = \frac{D}{1+\log_2(\max(2,|V|))}.
\]
正規化された複雑性は
\[
C_{\text{norm}} = H_{\text{norm}}\cdot R \cdot D_{\text{norm}}
\]
となる。これにより \(C_{\text{norm}}\in[0,\text{some small}]\) で比較可能になる。
```
